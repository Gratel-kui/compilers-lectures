% overview.tex

%%%%%%%%%%%%%%%%%%%%
\begin{frame}{}
  \begin{center}
    \blue{\Large ``father of the analysis of algorithms''}
  \end{center}

  \begin{columns}
    \column{0.50\textwidth}
      \fig{width = 0.70\textwidth}{figs/knuth-chair}
      \begin{center}
        \teal{Donald E. Knuth (1938 $\sim$)}
      \end{center}
    \column{0.50\textwidth}
      \fig{width = 0.80\textwidth}{figs/taocp-box}
  \end{columns}
\end{frame}
%%%%%%%%%%%%%%%%%%%%

%%%%%%%%%%%%%%%%%%%%
\begin{frame}{}
  \fig{width = 0.30\textwidth}{figs/selected-papers-lang}

  \begin{center}
    \red{ALGO Compiler \qquad {\bf LR Parser} \qquad {\bf Attribute Grammar}}
  \end{center}
\end{frame}
%%%%%%%%%%%%%%%%%%%%

%%%%%%%%%%%%%%%%%%%%
\begin{frame}{}
  \begin{center}
    \fig{width = 0.40\textwidth}{figs/video}

    \vspace{0.50cm}
    \href{https://youtu.be/QeiuVNDQg4k}{Writing a Compiler for the Burroughs Corporation}
  \end{center}
\end{frame}
%%%%%%%%%%%%%%%%%%%%

%%%%%%%%%%%%%%%%%%%%
\begin{frame}{}
  \begin{center}
    \[
      \text{``高级''语言 $\implies$ ``低级''语言 (如, 汇编语言)}
    \]
    \fig{width = 0.40\textwidth}{figs/compiler-blackbox}
  \end{center}
\end{frame}
%%%%%%%%%%%%%%%%%%%%

%%%%%%%%%%%%%%%%%%%%
\begin{frame}{}
  \begin{center}
    {\large 两个月的``编译器设计原理''之旅}

    \fig{width = 0.40\textwidth}{figs/keep-calm-open-box}
  \end{center}
\end{frame}
%%%%%%%%%%%%%%%%%%%%

%%%%%%%%%%%%%%%%%%%%
% \begin{frame}[fragile]{}
%   \begin{columns}
%     \column{0.60\textwidth}
%       \fig{width = 0.80\textwidth}{figs/lang-processing-system}
%     \column{0.40\textwidth}
%       \fig{width = 0.80\textwidth}{figs/gcc-v}
%   \end{columns}
% \end{frame}
%%%%%%%%%%%%%%%%%%%%

%%%%%%%%%%%%%%%%%%%%
\begin{frame}{}
  \begin{center}
    \fig{width = 0.80\textwidth}{figs/front-back}

    \vspace{0.50cm}
    IR: Intermediate Representation (中间表示)

    \vspace{0.50cm}
    模块化设计
  \end{center}
\end{frame}
%%%%%%%%%%%%%%%%%%%%

%%%%%%%%%%%%%%%%%%%%
\begin{frame}{}
  \begin{center}
    \fig{width = 0.80\textwidth}{figs/optimizer}

    \vspace{0.50cm}
    机器无关的中间表示优化
  \end{center}
\end{frame}
%%%%%%%%%%%%%%%%%%%%

%%%%%%%%%%%%%%%%%%%%
\begin{frame}{}
  \begin{columns}
    \column{0.50\textwidth}
      \begin{center}
        编译器前端: 分析阶段
      \end{center}
      \fig{width = 0.95\textwidth}{figs/front-end}
    \column{0.50\textwidth}
      \begin{center}
        编译器后端: 综合阶段
      \end{center}
      \fig{width = 0.50\textwidth}{figs/back-end}
  \end{columns}
\end{frame}
%%%%%%%%%%%%%%%%%%%%

%%%%%%%%%%%%%%%%%%%%
\begin{frame}{}
  \begin{center}
    \red{\bf 词法分析器 (Scanner):} 将{\bf 字符}流转化为{\bf 词法单元} (token) 流。
  \end{center}

  \begin{exampleblock}{词法分析 (Lexical Analysis)}
    \begin{center}
      \fig{width = 0.50\textwidth}{figs/assignment}
      \[
        \text{token}: \langle \text{token-class}, \text{attribute-value} \rangle
      \]
      \begin{align*}
        \langle \id, \red{1} \rangle \quad 
        \langle \ws \rangle \quad
        \langle \assign \rangle \quad
        \langle \ws \rangle \quad
        \langle \id, \red{2} \rangle \quad
        \langle \ws \rangle \quad \\
        \langle + \rangle \quad
        \langle \ws \rangle \quad
        \langle \id, \red{3} \rangle \quad
        \langle \ws \rangle \quad
        \langle \ast \rangle \quad
        \langle \ws \rangle \quad
        \langle \num, \red{4} \rangle
      \end{align*}
    \end{center}
  \end{exampleblock}
\end{frame}
%%%%%%%%%%%%%%%%%%%%

%%%%%%%%%%%%%%%%%%%%
\begin{frame}{}
  \begin{center}
    \red{\bf 语法分析器 (Parser):} 构建{\bf 词法单元}之间的{\bf 语法结构}
  \end{center}

  \begin{exampleblock}{语法分析 (Syntax Analysis)}
    \begin{center}
      % \fig{width = 0.50\textwidth}{figs/assignment}
      \fig{width = 0.60\textwidth}{figs/syntax-tree}
    \end{center}
  \end{exampleblock}
\end{frame}
%%%%%%%%%%%%%%%%%%%%

%%%%%%%%%%%%%%%%%%%%
\begin{frame}{}
  \begin{center}
    \red{\bf 语义分析器:} 语义检查, 如类型检查、一致性约束检查
  \end{center}

  \begin{exampleblock}{语义分析 (Semantic Analysis)}
    \begin{center}
      % \fig{width = 0.50\textwidth}{figs/assignment}
      \fig{width = 0.50\textwidth}{figs/semantic-analysis}
    \end{center}
  \end{exampleblock}
\end{frame}
%%%%%%%%%%%%%%%%%%%%

%%%%%%%%%%%%%%%%%%%%
\begin{frame}{}
  \begin{center}
    \red{\bf 中间代码生成器:} 生成中间代码, 如 {\bf ``三地址代码''}
  \end{center}

  \begin{exampleblock}{中间代码生成}
    \begin{center}
      % \fig{width = 0.50\textwidth}{figs/assignment}
      \fig{width = 0.50\textwidth}{figs/ICG}
    \end{center}
  \end{exampleblock}
\end{frame}
%%%%%%%%%%%%%%%%%%%%%%%%%%%%%%%%%%%%%%%

%%%%%%%%%%%%%%%%%%%%
\begin{frame}{}
  \begin{exampleblock}{中间代码优化}
    \begin{center}
      % \fig{width = 0.50\textwidth}{figs/assignment}
      \fig{width = 0.50\textwidth}{figs/CO}
    \end{center}
  \end{exampleblock}
\end{frame}
%%%%%%%%%%%%%%%%%%%%%%%%%%%%%%%%%%%%%%%

%%%%%%%%%%%%%%%%%%%%
\begin{frame}{}
  \begin{center}
    \red{\bf 代码生成器:} 生成目标代码, 主要任务包括{\bf 指令选择、寄存器分配}
  \end{center}

  \begin{exampleblock}{中间代码生成}
    \begin{center}
      % \fig{width = 0.50\textwidth}{figs/assignment}
      \fig{width = 0.50\textwidth}{figs/CG}
    \end{center}
  \end{exampleblock}
\end{frame}
%%%%%%%%%%%%%%%%%%%%%%%%%%%%%%%%%%%%%%%

%%%%%%%%%%%%%%%%%%%%%%%%%%%%%%%%%%%%%%%
\begin{frame}{}
  \begin{center}
    \red{\bf 符号表:} 收集并管理{\bf 变量名/函数名}相关的信息
  \end{center}

  \fig{width = 0.40\textwidth}{figs/symbol-table}
\end{frame}
%%%%%%%%%%%%%%%%%%%%%%%%%%%%%%%%%%%%%%%

%%%%%%%%%%%%%%%%%%%%%%%%%%%%%%%%%%%%%%%
\begin{frame}{}
  \fig{width = 0.80\textwidth}{figs/st-api}

  \vspace{0.50cm}
  \begin{center}
    红黑树 (RB-Tree)、哈希表 (Hashtable)
  \end{center}
\end{frame}
%%%%%%%%%%%%%%%%%%%%%%%%%%%%%%%%%%%%%%%

%%%%%%%%%%%%%%%%%%%%%%%%%%%%%%%%%%%%%%%
\begin{frame}{}
  \begin{center}
    符号表可能有多个
  \end{center}

  \fig{width = 0.50\textwidth}{figs/nested-symbol-tables}
\end{frame}
%%%%%%%%%%%%%%%%%%%%%%%%%%%%%%%%%%%%%%%