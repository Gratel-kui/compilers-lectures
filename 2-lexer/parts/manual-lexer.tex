% manual-lexer.tex

%%%%%%%%%%%%%%%%%%%%
\begin{frame}{}
  \fig{width = 0.60\textwidth}{figs/by-hand}

  \begin{center}
    手写词法分析器
  \end{center}
\end{frame}
%%%%%%%%%%%%%%%%%%%%

%%%%%%%%%%%%%%%%%%%%
\begin{frame}{}
  \begin{center}
    识别字符串$s$中符合{\bf 某种词法单元模式}的{\bf 所有词素}
    \fig{width = 0.50\textwidth}{figs/program}
    \ws \quad \ifkw \quad \elsekw \quad \id \quad \blue{\intnum \quad \floatnum \quad \scinum}
    \quad \relop \quad \assign\; (\texttt{:=})

    \pause
    \vspace{0.80cm}
    识别字符串$s$中符合{\bf 某种词法单元模式}的\red{\bf 前缀词素}

    \pause
    \vspace{0.80cm}
    识别字符串$s$中符合\red{\bf 特定词法单元模式}的{\bf 前缀词素}
  \end{center}
\end{frame}
%%%%%%%%%%%%%%%%%%%%

%%%%%%%%%%%%%%%%%%%%
\begin{frame}{}
  \begin{center}
    识别字符串$s$中符合\red{\bf 特定词法单元模式}的{\bf 前缀词素}

    \vspace{0.80cm}
    \blue{\fbox{\red{\bf 分支:} 先判断属于哪一类, 然后进入特定词法单元的前缀词素匹配流程}}

    \vspace{0.80cm}
    识别字符串$s$中符合{\bf 某种词法单元模式}的\red{\bf 前缀词素}

    \vspace{0.80cm}
    \blue{\fbox{\red{\bf 循环:} 返回当前识别出来的词法单元与词素, 继续识别下一个前缀词素}}

    \vspace{0.80cm}
    识别字符串$s$中符合{\bf 某种词法单元模式}的{\bf 所有词素}
  \end{center}
\end{frame}
%%%%%%%%%%%%%%%%%%%%

%%%%%%%%%%%%%%%%%%%%
\begin{frame}{}
  \begin{center}
    先: \ws \quad \ifkw \quad \elsekw \quad \id \quad \intnum 

    \vspace{1.00cm}
    然后: \relop 
    
    \vspace{1.00cm}
    最后: \floatnum \quad \scinum

    \vspace{1.00cm}
    \gray{留给大家: \assign\; (\texttt{:=})}
  \end{center}
\end{frame}
%%%%%%%%%%%%%%%%%%%%

%%%%%%%%%%%%%%%%%%%%
\begin{frame}{}
  \begin{center}
    识别字符串$s$中符合\red{\bf 特定词法单元模式}的{\bf 前缀词素}

    \vspace{0.20cm}
    \fig{width = 0.80\textwidth}{figs/peek-def}

    \vspace{0.30cm}
    \begin{columns}
      \column{0.10\textwidth}
      \column{0.80\textwidth}
        \begin{description}
          \setlength{\itemsep}{8pt}
          \item[line:] 行号, 用于调试
          \item[peek:] \red{\bf 下一个向前看字符 (Lookahead)}
          \item[words:] 从词素到词法单元\blue{\bf 标识符或关键词}的映射表 \\[5pt]
            \texttt{words.put("if", \ifkw)} \\
            \texttt{words.put("else", \elsekw)}
        \end{description}
      \column{0.10\textwidth}
    \end{columns}
  \end{center}
\end{frame}
%%%%%%%%%%%%%%%%%%%%

%%%%%%%%%%%%%%%%%%%%
\begin{frame}{}
  \begin{center}
    识别字符串$s$中符合\red{\bf 特定词法单元模式}的{\bf 前缀词素}

    \vspace{0.50cm}
    \ws: \blank \quad \tab \quad \line

    \pause
    \vspace{0.50cm}
    \fig{width = 0.80\textwidth}{figs/ws-code}

    识别空白部分, 并忽略之

    \pause
    \vspace{0.50cm}
    \fig{width = 0.60\textwidth}{figs/ws-code-wrong}
    \red{$Q:$ 这样写, 可不可以?}
  \end{center}
\end{frame}
%%%%%%%%%%%%%%%%%%%%

%%%%%%%%%%%%%%%%%%%%
\begin{frame}{}
  \begin{center}
    识别字符串$s$中符合\red{\bf 特定词法单元模式}的{\bf 前缀词素}

    \vspace{0.60cm}
    \ws: \blank \quad \tab \quad \line

    \vspace{0.60cm}
    \fig{width = 0.50\textwidth}{figs/ws}

    用于识别\blue{空白符}的状态转移图

    \pause
    \vspace{0.60cm}
    \red{$\ast$:} 识别出的空白符\blue{\bf 不包含}当前\texttt{peek}指向的字符
    
    \pause
    \vspace{0.40cm}
    \red{$22:$} 碰到\texttt{other}怎么办?
  \end{center}
\end{frame}
%%%%%%%%%%%%%%%%%%%%

%%%%%%%%%%%%%%%%%%%%
\begin{frame}{}
  \begin{center}
    识别字符串$s$中符合\red{\bf 特定词法单元模式}的{\bf 前缀词素}

    \vspace{0.60cm}
    \intnum: 整数 (简化处理, 允许以0开头)

    \pause
    \vspace{0.60cm}
    \fig{width = 0.80\textwidth}{figs/intnum-code}
  \end{center}
\end{frame}
%%%%%%%%%%%%%%%%%%%%

%%%%%%%%%%%%%%%%%%%%
\begin{frame}{}
  \begin{center}
    识别字符串$s$中符合\red{\bf 特定词法单元模式}的{\bf 前缀词素}

    \vspace{0.60cm}
    \intnum: 整数 (简化处理, 允许以0开头)

    \vspace{0.60cm}
    \fig{width = 0.40\textwidth}{figs/intnum}
    
    用于识别\blue{\intnum}的状态转移图

    \pause
    \vspace{0.40cm}
    \red{$12:$} 碰到\texttt{other}怎么办?
  \end{center}
\end{frame}
%%%%%%%%%%%%%%%%%%%%

%%%%%%%%%%%%%%%%%%%%
\begin{frame}{}
  \begin{center}
    识别字符串$s$中符合\red{\bf 特定词法单元模式}的{\bf 前缀词素}

    \vspace{0.50cm}
    \id: 字母开头的字母/数字串

    \pause
    \vspace{0.20cm}
    \fig{width = 0.80\textwidth}{figs/id-code}

    \blue{识别词素、\red{\bf 判断是否是预留的关键字或已识别的标识符}、保存该标识符}
  \end{center}
\end{frame}
%%%%%%%%%%%%%%%%%%%%

%%%%%%%%%%%%%%%%%%%%
\begin{frame}{}
  \begin{center}
    识别字符串$s$中符合\red{\bf 特定词法单元模式}的{\bf 前缀词素}

    \vspace{0.60cm}
    \id: 字母开头的字母/数字串

    \vspace{0.60cm}
    \fig{width = 0.80\textwidth}{figs/id}

    \pause
    \vspace{0.40cm}
    \red{$9:$} 碰到\texttt{other}怎么办?
  \end{center}
\end{frame}
%%%%%%%%%%%%%%%%%%%%

%%%%%%%%%%%%%%%%%%%%
\begin{frame}{}
  \begin{center}
    识别字符串$s$中符合\red{\bf 特定词法单元模式}的{\bf 前缀词素}

    \vspace{0.60cm}
    \fig{width = 0.60\textwidth}{figs/other-code}

    \vspace{0.50cm}
    错误处理模块: 出现\red{\bf 词法错误}, 直接报告异常字符
  \end{center}
\end{frame}
%%%%%%%%%%%%%%%%%%%%

%%%%%%%%%%%%%%%%%%%%
\begin{frame}{}
  \begin{center}
    识别字符串$s$中符合\red{\bf 某种词法单元模式}的{\bf 前缀词素} 
    \blue{($\Call{scan}{\null}$)}
  \end{center}

  \begin{columns}
    \column{0.45\textwidth}
      \begin{center}
        \fig{width = 0.80\textwidth}{figs/ws}
        \ws: 空白符

        \vspace{0.20cm}
        \fig{width = 0.60\textwidth}{figs/intnum}
        \intnum: 整数
      \end{center}
    \column{0.55\textwidth}
      \begin{center}
        \fig{width = 1.00\textwidth}{figs/id}
        \id: 标识符

        \vspace{0.40cm}
        \fig{width = 0.80\textwidth}{figs/other-code}
        错误处理模块
      \end{center}
  \end{columns}

  \pause
  \vspace{0.30cm}
  \begin{center}
    \red{\bf 关键点:} 合并$22, 12, 9$, 根据\purple{\bf 下一个字符}即可判定词法单元的类型 \\[8pt]
    否则, 调用错误处理模块(对应\texttt{other}), 报告\purple{\bf 该字符有误}, 并忽略该字符
  \end{center}
\end{frame}
%%%%%%%%%%%%%%%%%%%%

%%%%%%%%%%%%%%%%%%%%
\begin{frame}{}
  \fig{width = 0.80\textwidth}{figs/code-1}
\end{frame}
%%%%%%%%%%%%%%%%%%%%

%%%%%%%%%%%%%%%%%%%%
\begin{frame}{}
  \fig{width = 0.65\textwidth}{figs/code-2}
\end{frame}
%%%%%%%%%%%%%%%%%%%%

%%%%%%%%%%%%%%%%%%%%
\begin{frame}{}
  \begin{center}
    识别字符串$s$中符合{\bf 某种词法单元模式}的\red{\bf 所有词素}

    \vspace{1.00cm}
    外层\blue{\bf 循环}调用 \blue{$\Call{scan}{\null}$}

    \vspace{0.60cm}
    或者, 由语法分析器\red{\bf 按需}调用 \blue{$\Call{scan}{\null}$}

    \vspace{0.20cm}
    \fig{width = 0.50\textwidth}{figs/lexer}
  \end{center}
\end{frame}
%%%%%%%%%%%%%%%%%%%%

%%%%%%%%%%%%%%%%%%%%
\begin{frame}{}
  \begin{center}
    \fig{width = 0.80\textwidth}{figs/ws-code}
    \fig{width = 0.60\textwidth}{figs/ws-code-wrong}

    \pause
    \texttt{char peek = ''}: 下一个向前看字符

    \pause
    \vspace{0.60cm}
    \blue{\bf 考虑例子 ``123abc''}

    \pause
    \vspace{-0.50cm}
    \[ 
      \langle \num, 123 \rangle \qquad \langle \id, abc \rangle
    \]
    % \red{\bf 循环不变式:} \\[6pt]
    % 当\textsc{\blue{scan()}} 返回一个词法单元时, \\[3pt]
    % \texttt{peek} 是空白符或者是\blue{当前词素后的第一个字符}
  \end{center}
\end{frame}
%%%%%%%%%%%%%%%%%%%%

%%%%%%%%%%%%%%%%%%%%
\begin{frame}{}
  \begin{center}
    识别字符串$s$中符合\red{\bf 特定词法单元模式}的{\bf 前缀词素}

    \vspace{0.30cm}
    \relop~\footnote{此处,\texttt{=}是判断是否相等的关系运算符。
    请考虑, 如果\texttt{=}表示赋值, \texttt{==} 表示相等判断, 该如何设计词法分析器?}: 
    \texttt{< \quad > \quad <= \quad >= \quad \red{=} \quad <>}

    \pause
    \vspace{0.30cm}
    \fig{width = 0.50\textwidth}{figs/relop}

    \pause
    \blue{\bf ``最长优先原则'':} 例如, 识别出\texttt{<=}, 而不是\texttt{<}与\texttt{=}

    \pause
    \red{$0:$} 碰到\texttt{other}怎么办?
  \end{center}
\end{frame}
%%%%%%%%%%%%%%%%%%%%

%%%%%%%%%%%%%%%%%%%%
\begin{frame}{}
  \begin{center}
    识别字符串$s$中符合\red{\bf 某种词法单元模式}的{\bf 前缀词素} 
    \blue{($\Call{scan}{\null}$)}
  \end{center}

  \begin{columns}
    \column{0.45\textwidth}
      \begin{center}
        \fig{width = 0.80\textwidth}{figs/ws}
        \fig{width = 0.60\textwidth}{figs/intnum}
        \fig{width = 1.00\textwidth}{figs/id}
      \end{center}
    \column{0.55\textwidth}
      \begin{center}
        \fig{width = 0.70\textwidth}{figs/relop}
        \fig{width = 0.80\textwidth}{figs/other-code}
      \end{center}
  \end{columns}

  \pause
  \vspace{0.30cm}
  \begin{center}
    \red{\bf 关键点:} 合并$22, 12, 9, 0$, 根据\purple{\bf 下一个字符}即可判定词法单元的类型 \\[8pt]
    否则, 调用错误处理模块(对应\texttt{other}), 报告\purple{\bf 该字符有误}, 并忽略该字符
  \end{center}
\end{frame}
%%%%%%%%%%%%%%%%%%%%

%%%%%%%%%%%%%%%%%%%%
\begin{frame}{}
  \begin{center}
    但是, 词法分析器的设计并没有这么容易

    \vspace{0.50cm}
    \fig{width = 0.50\textwidth}{figs/life-always-hard}
  \end{center}
\end{frame}
%%%%%%%%%%%%%%%%%%%%

%%%%%%%%%%%%%%%%%%%%
\begin{frame}{}
  \begin{center}
    \ws \quad \ifkw \quad \elsekw \quad \id \quad \intnum \quad \relop 
    
    \vspace{0.80cm}
    根据\purple{\bf 下一个字符}即可判定词法单元的类型

    \vspace{0.50cm}
    每个状态转移图的每个状态要么是\purple{\bf 接受状态}, 要么带有\purple{\bf \texttt{other}边}

    \pause
    \vspace{1.20cm}
    \red{如何同时识别\; \floatnum{} 与 \scinum?}
  \end{center}
\end{frame}
%%%%%%%%%%%%%%%%%%%%

%%%%%%%%%%%%%%%%%%%%
\begin{frame}{}
  \begin{columns}
    \column{0.50\textwidth}
      \fig{width = 0.70\textwidth}{figs/intnum}
      \begin{center}
        \intnum: 整数
      \end{center}
    \column{0.50\textwidth}
      \pause
      \fig{width = 1.00\textwidth}{figs/realnum}
      \begin{center}
        \floatnum: 浮点数 (无科学计数法)\\
        (不识别 \texttt{2.})
      \end{center}
  \end{columns}

  \pause
  \vspace{1.00cm}
  \fig{width = 0.80\textwidth}{figs/scinum}
  \begin{center}
    \scinum: 带科学计数法的浮点数 \\
    (\texttt{2.99792458E8 \quad 3E8})
  \end{center}
\end{frame}
%%%%%%%%%%%%%%%%%%%%

%%%%%%%%%%%%%%%%%%%%
\begin{frame}{}
  \begin{center}
    \num: 
    整数部分\blue{[\texttt{.}可选的小数部分]}\purple{[\texttt{E}[可选的\texttt{+-}]可选的指数部分]}

    \pause
    \vspace{0.60cm}
    \fig{width = 0.80\textwidth}{figs/number}
    \red{$19, 20, 21:$} 代表了不同的数字类型 

    \pause
    \vspace{0.80cm}
    \red{$12:$} 碰到\texttt{other}怎么办?
    \pause {\hfill \small (尝试其它词法单元或进入错误处理模块)}

    \pause
    \vspace{0.30cm}
    \red{$14, 16, 17:$} 碰到\texttt{other}怎么办?
    \pause
    {\hfill (回退, 寻找\red{\bf 最长匹配})}
  \end{center}
\end{frame}
%%%%%%%%%%%%%%%%%%%%

%%%%%%%%%%%%%%%%%%%%
\begin{frame}{}
  \begin{center}
    识别字符串$s$中符合\red{\bf 某种词法单元模式}的{\bf 前缀词素} 
    \blue{($\Call{scan}{\null}$)}
  \end{center}

  \begin{columns}
    \column{0.50\textwidth}
      \begin{center}
        \fig{width = 0.80\textwidth}{figs/ws}
        \fig{width = 1.00\textwidth}{figs/number}
        \fig{width = 1.00\textwidth}{figs/id}
      \end{center}
    \column{0.50\textwidth}
      \begin{center}
        \fig{width = 0.70\textwidth}{figs/relop}
        \fig{width = 0.80\textwidth}{figs/other-code}
      \end{center}
  \end{columns}

  \pause
  \vspace{0.30cm}
  \begin{center}
    \red{\bf 关键点:} 合并$22, 12, 9, 0$, 根据\purple{\bf 下一个字符}即可判定词法单元的类型 \\[4pt]
    否则, 调用错误处理模块(对应\texttt{other}), 报告\purple{\bf 该字符有误}, 忽略该字符。 \\[4pt]
    注意, 在\scinum{}中, 有时需要\purple{\bf 回退}, 寻找最长匹配。
  \end{center}
\end{frame}
%%%%%%%%%%%%%%%%%%%%